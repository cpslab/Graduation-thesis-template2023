%%%%%%%%%%%% ------------ 2 page ------------  %%%%%%%%%%%%
\newpage
\pagestyle{plain}

\begin{flushleft}
{\huge{\bf 謝辞}}\\
\vspace{1cm}
% 感謝を述べましょう^^

\vspace{3cm}
\begin{flushright}
20xx年x月xx日\\
電大 太郎\\
\end{flushright}
\end{flushleft}

% \addcontentsline{toc}{chapter}{}
\newpage
\begin{flushleft}
{\huge{\bf 学外発表}}\\
\vspace{1cm}
\begin{enumerate}
% enumerate 数字付き箇条書き
\item \underline{電大太郎},岩井 将行,"タイトル","学会名",PaperID=???,20xx年x月発表予定

\end{enumerate}
\end{flushleft}

% 参考文献
\renewcommand{\bibname}{参考文献}
\bibliography{Ref}
\bibliographystyle{junsrt}


\chapter*{付録}\label{huroku}
\addcontentsline{toc}{chapter}{付録}
\begin{flushleft}
付録として,〇〇学会に投稿した論文と,実験の様子を収めた写真を添付する.
\end{flushleft}

\newpage

\newgeometry{left=2cm,bottom=2cm,top=2cm}
\pagestyle{plain}

% 学会用のpdfを添付する
% \includepdf[pages=-]{フォルダ/ファイル名.pdf}

\newpage
\restoregeometry

% 写真↓
% \begin{center}
%   \begin{figure}
%     % \center
%     \includegraphics[width=14cm]{appendix/yousu.png}
%     \caption{実験の様子}
%     \label{ex}
%   \end{figure}
% \end{center}
% 参考:https://medemanabu.net/latex/graphicx-figure-includegraphics/
