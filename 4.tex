%%%%%%%%%%%% ------------ chapter header ------------  %%%%%%%%%%%%
\chapterhead
% chapter number, title
{実装}
% chapter abst
{本章では,〜〜について述べる.}

%%%%%%%%%%%% ------------ next page ------------  %%%%%%%%%%%%

\section{実装}
ここに文章を入れる.
% 長い場合は\input{texファイル名}で挿入など工夫する
% 図1
% 表(てきとう)
\begin{table}[h]
  \begin{center}
    \caption{システムAとシステムBの比較}
    \begin{tabular}{|l|l|l|} \hline
         システム   & A & B \\ \hline
         速さ          & 1 & 2 \\ \hline
         手軽さ       & 3 & 4 \\ \hline
         持ちやすさ & 5 & 6 \\ \hline
         重さ          & 7 & 8 \\ \hline
    \end{tabular}
    \label{tb:angle}
  \end{center}
\end{table}
% 参考:https://atatat.hatenablog.com/entry/cloud_latex7_table
% 位置
% []の中身
% h figure環境宣言した場所(here)に図の場所を確保
% t figure環境宣言したページの上部(top)に図の場所を確保
% b figure環境宣言したページの下部(bottom)に図の場所を確保
% p 最後のページ(p)に図の場所を確保


% 数式
% $x = 1$ のように書く
\newpage

