%%%%%%%%%%%% ------------ chapter header ------------  %%%%%%%%%%%%
\chapterhead
% chapter number, title
{序論}
% chapter abst
{本章では,本研究の背景と目的および,本論文の内容構成について述べる.}

%%%%%%%%%%%% ------------ next page ------------  %%%%%%%%%%%%

% \section{}で小見出しを作る
%  章ごとにファイルを分けている

\section{背景}
ここに文章を入れる.\par
% \parで段落を変える
次の点に気をつける.
% 以下箇条書きの例
\begin{itemize}
  \item ”,”と”,” ”.”と”.”の違いに気をつける.
  \item 逆スラッシュはoptionを押しながら¥を入力すると出る(mac).
  \item 表記,特に英数字を気をつける.(M5Stack,m5stackなど)
  \item 文末は揃える.
\end{itemize}

% \cite{タグ名}で引用 参考文献の欄に詳細を書く
% \newpageでページ切り替え

%%%%%%%%%%%% ------------ next page ------------  %%%%%%%%%%%%

\section{本研究の目的}
本研究では,〜〜システムを提案する.

\newpage
\section{本論文の構成}
本論文の構成は以下の通りである.
第2章では,〜〜について述べる.第3章では,〜〜について述べる.第4章では〜〜について述べる.第5章では,〜〜述べ,考察を行う.第6章では,本論文の結論を述べる.
\newpage